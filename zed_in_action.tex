\vspace{-0.6em}
\section{Data Processing with \sys{}} \label{s:zed_in_action}

Here we describe a \sys{}-based data-processing ecosystem, and how \sys{} overcomes the challenges of hybrid approaches (\S\ref{s:hybrid}).

\vspace{-0.9em}
\subsection{Data Generation and Ingestion with \sys{}}

Data sources may generate data in a number of common formats (e.g., JSON, CSV, Parquet, Avro, etc.) or if the source is \sys{}-aware, any of the native \sys{} formats can be used as well.  While Zed's command-line tooling can directly parse, search, and analyze files stored directly on the file system, \zed{} data can also be organized into data "pools" that comprise a \sys{} data lake.  The \zed{} service manages lake data in accordance with the \sys{} lake format allowing concurrent ingestion following a Git-like branching and commit history, time travel, and scale-out query performance.  To do so, \sys{} employs transactional reads and writes using an immutable transaction log, as in Iceberg or Delta Lake~\cite{delta_lake}, but the transaction history resembles Git's design instead of a monolithic log.  The \sys{} lake portion of \sys{} is under ongoing development and is currently used in production by a number of early-adopter data teams.

The \textt{zui} application can be used to search, browse, and analyze \sys{} lake data, or an API to the lake can be used programmatically by Go or Python clients with support for other languages forthcoming.

Compared to generating and ingesting data in the document model, using \sys{} is no more difficult.  When clients use native \sys{} formats, the full richness of the Zed data model can be exploited, e.g., defining named types to faciliate the relationship between specific data structures in a native programming language and the named type in Zed.  While named type definitions require specifying a name for the type as well as all the underlying type (often a record type comprising field names and field types), this is typically not burdensome for data sources as the data they generate corresponds to data structures that are already defined in memory.

Specifically, in our Go implementation, marshaling functions can be configured to automatically use a Zed type name that is derived from the Go type name of the data being serialized.  This allows the serialized data later to be easily unmarshaled into a Go interface value, where the underlying concrete type is automatically inferred from the Zed type name.  There is no need to write complicated unmarshaling logic as required by JSON deserializers when different concrete types can all implement the same interface value.

Zed clients may also generate Zed data via the "builder pattern" where complex Zed values may be constructed with calls to a Zed client library, which includes the ability to create named types. When a source wants to define a new type or redefine a named type, it simply starts writing data with that new type, as shown in Figure~\ref{f:iot_data}, and as described earlier, a named type comprising more than one distinct underlying type is automatically converted to a sum type as the Zed system processes a stream of values.

Of course, data sources may forgo named types altogether and this is very common; in this case writing \sys{} data is very similar to writing JSON.

When \sys{} data arrives at the storage layer in a Zed lake, it can store be stored as both ZNG or VNG, and can also build indexes over it. How the data is split across different formats may impact query performance but has no impact on the types of queries that can be issued, the way they are expressed in the query language, or what the results look like. 

Ingesting \sys{} data is much easier than ingesting data into relational or hybrid systems. Users might choose to transform data on ingest, but they are never required to predefine schemas or to clean data to conform to any particular structure before it can be stored.

Users who are unable to modify their data sources to output data in one of \sys{}'s native formats can still benefit from \sys{}.  As mentioned, above \sys{} can consume data in a number of popular formats and automatically convert the input to native Zed formats on the fly as well as convert outputs of queries to any desired format.  As such, users can query data in any supported format using \sys{}'s query language and convert the output as desired incurring a performance cost to perform format conversion. In addition, if users would like to {\em shape} their data into a specific set of data types, \sys{}'s type abstraction and extensive support for type conversions and data transformations all provide an excellent foundation compared to in existing systems; we leave an in-depth exploration of this shaping functionality to future work.

%Users can use \sys{}'s support for introspection (\S\ref{ss:zed_in_action_types}) to view the types of data present in their dataset. Finally, \sys{}'s query language provides operators to help {\em shape} data 

\begin{figure}
    \begin{minted}[]{bash}
{
    time: 18:47:00 (time),
    triggered: true (bool),
    loc: {floor: 3 (int32), room: "robotics lab" (string)} (=location)
} (=motion)
{
    time: 19:03:00 (time),
    triggered: true (bool),
    loc: {floor: 2 (int32), room: "kitchen" (string)} (=location)
} (=motion)
{
    time: 19:04:00 (time),
    volume: 9.5 (float32),
    loc: {floor: 4 (int32), room: "media lab" (string)} (=location)
} (=sound)
    \end{minted}
    \vspace{-1.3em}
    \caption{Nested IoT data output by motion and sound sensors in different rooms, represented in ZSON.}
    \label{f:iot_data_nested}
    \vspace{-1.5em}
\end{figure}

\vspace{-0.9em}
\subsection{Querying Data in \sys{}} \label{ss:zed_querying}

All \sys{} data, regardless of the format that it is stored in, can benefit from both the document model's flexibility and the relational model's types. \sys{} supports queries that are typically issued over data in the document model today, because they involve mixing heterogeneous data. For example, \sys{} supports the query from Section~\ref{ss:hybrid_querying}, which sorts heterogeneously typed values by time and returns the first five. We can issue this query using the command-line tool \texttt{zq}:

\texttt{zq "sort time | head 5" sensor\_data.zng}

\noindent{}Figure~\ref{f:iot_data} shows example results for this query. \sys{} can also search data, for example performing a full-text search over the nested data shown in Figure~\ref{f:iot_data_nested}:

\texttt{zq "search `lab'" nested\_sensor\_data.zng}

\noindent{}This query searches all string fields in the heterogeneous data, including the nested \texttt{room} field, and returns the first and third records.

\sys{} can also leverage types for efficient analytics, as is commonly done with relational data today. For example, we can use issue an analytics query over the data shown in Figure~\ref{f:iot_data_nested}, formatted in columnar VNG:

\texttt{zq "count() by loc.floor" nested\_sensor\_data.vng}

\noindent{}This query counts the number of occurrences of each floor in the nested \texttt{location} records.

Though \sys{}'s formats can mix heterogeneous data, it's still possible to organize data separately by type, either within a file or by using multiple files. This enables all of the performance optimizations of existing relational systems, while providing the flexibility during ingestion and querying of document-based systems. Thus \sys{} subsumes both approaches at the same time.



\begin{figure}
    \begin{minted}[]{bash}
$ zq -f table 'count() by typeof(this)' sensor_data.zng
typeof                                           count
<temperature={time:time,temp:int32}>             452
<humidity={time:time,percent_humidity:float32}>  82
<temperature={time:time,temp:int32,unit:string}> 239
    \end{minted}
    \vspace{-1.3em}
    \caption{A data introspection query to count the number of records of each type in a ZNG file of IoT sensor data. \noteamy{fix this to use new proposal for unioning types instead of redefining them}}
    \label{f:count_by_type}
    \vspace{-1.5em}
\end{figure}

\vspace{-0.9em}
\subsection{Data Introspection in \sys{}} \label{ss:zed_in_action_types}

\sys{} enables two main forms of introspection. First, users can {\em query for types} to view what types are present in a dataset. Figure~\ref{f:count_by_type} shows an example of counting the number of values with each type in an IoT dataset and outputting the results in a tabular format (with \texttt{-f table}). Second, \sys{} supports {\em querying by type}. For example, querying for records with \texttt{typeof(this)==<humidity>} selects all records with type \texttt{humidity}, effectively extracting a relational \texttt{humidity} table out of a collection of heterogeneous records.
This enables \sys{} to provide the same functionality as relational systems, and to select groups of records with related types for further transformations or querying. To support type queries over automatically converted named sum types, the equality operator evaluates to true when a named sum type is compared with one of its constituent types.

These data introspection operations seem quite simple, but are actually quite powerful, especially when applied to large datasets with hundreds of complex nested data types. Because all \sys{} data is typed, these operations work at any stage of processing---when data is generated, during ingestion, while stored, or during querying---thereby avoiding the catch-22 of existing relational and hybrid approaches. Because type information is encoded by data sources, the number of distinct types is much smaller than if you treated each combination of JSON fields as a distinct type, and more accurate than if you relied on schema inference. And finally, because types are first-class members of both the query language and data model, all of these introspection queries reuse the same query language and data model that are used by regular queries over data (\S\ref{ss:zed_querying}).

Note that \sys{} does not eliminate the problem of data cleaning. As prior work has pointed out, a key part of data cleaning is resolving {\em semantic heterogeneity}, for example to determine if a \texttt{wages} field means the same thing in one dataset as another~\shortorlongform{\cite{what_goes_around, redbook}}{\cite{what_goes_around, redbook_integration}}. \sys{} still requires that users resolve such questions. However, \sys{}'s introspection capabilities can significantly ease data cleaning by providing visibility into the set of types present in a dataset. This enables users to focus on resolving the semantic heterogeneity between different types, rather than trying to infer which records correspond to the same type in the first place.


%The results of these queries, even the results in Figure~\ref{f:count_by_type} which include types, are presented in the \sys{} data model, because it supports first-class types as well.


\begin{comment}
Finally, \sys{} eases {\em shaping} by enabling users to shape values by type rather than field-by-field. \sys{} supports operators such as:
\begin{CompactItemize}
\item \texttt{crop(val, t)}: outputs \texttt{val} but omits any extra fields of \texttt{val} that are not in type \texttt{t}
\item \texttt{fill(val, t)}: outputs \texttt{val} but adds fields with a null value for all fields in type \texttt{t} that are not in \texttt{val}
\item \texttt{order(val, t)}: outputs \texttt{val} but with its fields re-ordered to match the order of fields in type \texttt{t}
\item \texttt{cast(val, t)}: outputs \texttt{val} but with the types of all its fields changed to match those of type \texttt{t}
\end{CompactItemize}

For example, suppose a user's temperature data is much more varied than what is shown in Figure~\ref{f:iot_data}, with many records containing 
%shaping: \texttt{crop(this, <{time:time,temp:int32,unit:string}>)}
\end{comment}

